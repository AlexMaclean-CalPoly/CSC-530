\documentclass{article}
\usepackage[utf8]{inputenc}
\usepackage[english]{babel}
\usepackage{amsthm, amssymb, amsmath, amsfonts, extramarks}
\usepackage{gen, color}


\title{Homework Assignment 1}
\author{Bailey Wickham \& Alex MacLean\\ CSC530}

\date\today

\newenvironment{homeworkProblem}[1]{
    \section*{Problem #1}
}

\newenvironment{solution}{\color{blue} \em }{}



\setlength\textwidth{6.5in}
\setlength\textheight{8.75in}
\setlength\oddsidemargin{0in}
\setlength\evensidemargin{1in}
\setlength\topmargin{-0.25in}
\setlength\parindent{0in}
\setlength\parskip{0.5em}

\begin{document}
\maketitle

\section{Propositional Logic and Normal Forms}


\begin{enumerate}
	\setcounter{enumi}{1}

\item (5 points)  Convert the following formula to an equisatisfiable one in CNF using Tseitin's encoding: 
\[\neg ( \neg r \rightarrow \neg (p \land q))\]
Write the final CNF as the answer.  
Use $a_\phi$ to denote the auxiliary variable for the formula $\phi$;
for example, $a_{p\wedge q}$ should be used to denote the auxiliary variable for $p\wedge q$.
Your conversion should not introduce auxiliary variables for negations.

\begin{solution}

Substitutions:
\begin{align*}
    a_{\phi} &\leftrightarrow \neg (\neg r \rightarrow \neg a_{p \wedge q}) \\
    a_{p\wedge q} &\leftrightarrow (p \wedge q)
\end{align*}
Conjunction:
\begin{gather*}
    a_{\phi} \wedge (a_{\phi} \leftrightarrow \neg (\neg r \rightarrow \neg a_{p \wedge q})) \land
    (a_{p\wedge q} \leftrightarrow (p \wedge q))
    \\ \text{ where }  \\
    (a_{\phi} \leftrightarrow \neg (\neg r \rightarrow \neg a_{p \wedge q})) \equiv
(\neg a_\phi \lor \neg r) \land (\neg a_\phi \lor a_{p \land q}) \land (r \lor \neg a_{p \land q} \lor a_\phi) \\
    \text{ and } \\
    (a_{p\wedge q} \leftrightarrow (p \wedge q)) \equiv
    (\neg a_{p\wedge q} \lor p ) \land (\neg a_{p\land q} \lor q) \wedge (\neg p \lor \neg q \lor a_{p\wedge q}) \\
    \text{ so } \\
    \phi \equiv
    a_\phi \land
(\neg a_\phi \lor \neg r) \land (\neg a_\phi \lor a_{p \land q}) \land (r \lor \neg a_{p \land q} \lor a_\phi) \land \\
    (\neg a_{p\wedge q} \lor p ) \land (\neg a_{p\land q} \lor q) \wedge (\neg p \lor \neg q \lor a_{p\wedge q})
\end{gather*}

\end{solution}

\item (10 points)  Let $\phi$ be a propositional formula in NNF, and let $I$ be an interpretation of $\phi$. Let the \emph{positive set} of $I$ with respect to $\phi$, denoted $\emph{pos}(I, \phi)$, be the literals of $\phi$ that are satisfied by $I$. As an example, for the NNF formula $\phi = (\neg r \land p) \lor q$ and the interpretation $I = [r \mapsto \bot, p \mapsto \top, q \mapsto \bot]$, we have $\emph{pos}(I, \phi) = \{ \neg r, p \}$. Prove the following theorem about the monotonicity of NNF: 

{\bf Monotonicity of NNF:} For every interpretation $I$ and $I'$ such that $\emph{pos}(I, \phi) \subseteq \emph{pos}(I', \phi)$, if $I \models \phi$, then $I' \models \phi$. 

(\textbf{Hint:} Use structural induction.) \label{prob:NNF}



\item (10 points) Let $\phi$ be an NNF formula.  Let $\hat{\phi}$ be a formula derived from $\phi$ using a modified version of Tseitin's encoding in which the CNF constraints are derived from implications rather than bi-implications.   For example, given the formula
\[a_1\land (a_2 \lor \neg a_3),\]
the new encoding is the CNF equivalent of the following, where $x_0, x_1, x_2$ are fresh auxiliary variables:\looseness=-1
\[
\begin{array}{ll}
x_0 & \land \\
(x_0 \rightarrow a_1 \land x_1) & \land \\
(x_1 \rightarrow a_2 \lor x_2) & \land \\
(x_2 \rightarrow \neg a_3) &  \\
\end{array}
\]
Note that Tseitin's encoding to CNF starts with the same formula, except that $\rightarrow$ is replaced with $\leftrightarrow$.  As a result, the new encoding has roughly half as many clauses as the Tseitin's encoding.

\medskip
Prove that $\hat{\phi}$ is satisfiable if and only if $\phi$ is satisfiable.  

\medskip
(\textbf{Hint}: Use the theorem from Problem \ref{prob:NNF}.)


\end{enumerate}

\end{document}

